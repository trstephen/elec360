\section{Results and discussion}\label{sec:results}
Sections \ref{sec:initial}, \ref{sec:mres}, and \ref{sec:mtorque} use static conditions to determine physical properties of the DC motor and parameters of its associated transfer function.
Section \ref{sec:bump} uses measured system input and output under dynamic conditions to determine the transfer function parameters.
The parameters derived from the prelab, static conditions and dynamic conditions are compared in Section \ref{sec:validation}.

\subsection{Initial experimental tests}\label{sec:initial}
The motor described in the lab manual \cite{lab-manual} is meant to be roughly equivalent to the motor used in the experiment.
One notable difference is that the lab manual motor has a maximum $u_m$ of \SI{15}{\volt} DC.
The QICii software used to control the experiment motor only allows a maximum $u_m$ of \SI{5}{\volt} DC.
To establish the validity of the lab manual motor parameters as approximations for the experiment motor we will compare the $\omega_{m\>max}$ from each source.

Using KVL in Figure~\ref{fig:motorschematic}, we can derive the relationship between the input voltage and the motor angular velocity as
\begin{equation}\label{eq:KVL}
  u_m(t) = R_m i_m(t) + L_m \pdv{i_m(t)}{t} + k_m \omega_m.
\end{equation}
Since $i_m(t) \approx \SI{0}{\ampere}$ at steady state, \eqref{eq:KVL} can be simplified to
\begin{equation*}
  \omega_{m\>max} = {u_{m\>max} \over k_m} = {\SI{15}{\volt} \over \SI{0.0502}{\volt\second\per\radian}} = \SI{298.8}{\radian\per\second}.
\end{equation*}
$\omega_{m\>max}$ can be obtained directly from the QICii software.
The response for $u_{m\>max} = \SI{5}{\volt}$ is shown in Figure~\ref{fig:wmax}.
\begin{figure}[t!]
  \centering
  \includegraphics[width=0.95\linewidth]{graphics/part2}
  \caption{Steady state response of DC motor to constant $u_m$}
  \label{fig:wmax}
\end{figure}

\pagebreak
For the experiment motor
\begin{equation*}
  k_m = {\SI{4.95}{\volt} \over \SI{94}{\radian\per\second}} = \SI{0.05266}{\volt\second\per\radian}
\end{equation*}
which is slightly larger than the $k_m$ in the lab manual.
Thus, the physical parameters in the lab manual are valid.
The difference between the two sets of parameters will be explored in Section~\ref{sec:validation}.

\subsection{Motor resistance}\label{sec:mres}
If $\omega_m = 0$ in \eqref{eq:KVL} then we can determine $R_m$ by measuring $i_m$.
This constraint is realized by holding the disc stationary while applying different $u_m$.
Note, a bias current, $i_{m\>bias} = \SI{-10}{\milli\ampere}$ was measured when $u_m = \SI{0}{\volt}$.
The $R_m$ in each case is derived from \eqref{eq:KVL} as
\begin{equation*}
  R_m = {u_m \over i_m - i_{m\>bias}}
\end{equation*}
Table~\ref{table:Rm} summarizes these results.
\begin{table}[htpb]
  \centering
  \caption{Motor voltage and current at steady state}
  \label{table:Rm}
  \begin{tabular}{@{}SSS@{}}
    \toprule
      \multicolumn{1}{c}{$u_m$ (\si{\volt})} &
      \multicolumn{1}{c}{$i_m$ (\si{\ampere})} &
      \multicolumn{1}{c}{$R_m$ (\si{\ohm})} \\
    \midrule
    -5.00 & -0.35 & 14.3 \\
    -2.00 & -0.15 & 13.3 \\
     1.00 &  0.05 & 20.0 \\
     2.00 &  0.13 & 15.4 \\
     5.00 &  0.33 & 15.2 \\
    \bottomrule
      \multicolumn{2}{r}{$R_{m\>avg}$} & 15.6 \\
  \end{tabular}
\end{table}

The $R_{m\>avg}$ is nearly 50\% larger than the value of \SI{10.6}{\ohm} used for calculations in the prelab.
\todo[inline]{Come up with a convincing explanation for this}

\subsection{Motor torque constant}\label{sec:mtorque}
Similar to the approach in Section~\ref{sec:mres}, we can determine $k_m$ by analyzing the oppopsite static condition of the motor.
When the motor is in a steady state of motion, $i_m \approx 0$.
Using \eqref{eq:KVL}, we can determine
\begin{equation*}
  k_m = {u_m \over \omega_m}.
\end{equation*}
Table~\ref{table:km} summarizes the results, with $u_m$ varied over the same range as Section~\ref{sec:mres}.
\begin{table}[htpb]
  \centering
  \caption{Motor angular velocity in a free-spinning motor}
  \label{table:km}
  \begin{tabular}{@{}SSS@{}}
    \toprule
      \multicolumn{1}{c}{$u_m$ (\si{\volt})} &
      \multicolumn{1}{c}{$\omega_m$ (\si{\radian\per\second})} &
      \multicolumn{1}{c}{$k_m$ (\si{\volt\second\per\radian})} \\
    \midrule
    -5.00 & -93 & 0.0538 \\
    -2.00 & -35 & 0.0571 \\
    1.00 & 16 & 0.0625 \\
    2.00 & 35 & 0.0571 \\
    5.00 & 92 & 0.0543 \\
    \bottomrule
      \multicolumn{2}{r}{$k_{m\>avg}$} & 0.0570 \\
  \end{tabular}
\end{table}

This is larger than the prelab value of $k_m = \SI{0.0502}{\volt\second\per\radian}$.
\todo[inline]{Come up with a convincing explanation for this}

\subsection{Open loop transfer function parameters from static analysis}\label{sec:tf-static}
Taking the Laplace transform of \eqref{eq:KVL} we can derive a first order transfer function of the form in \eqref{eq:tf}, where the open loop transfer function $G(s) = {\Omega_m(s) \over U_m(s)}$.
\begin{equation}\label{eq:KVL-tf}
  U_m(s) = R_m I_m + s L_m I_m + k_m \Omega_m
\end{equation}
\cite{lab-manual} tells us that
\begin{equation*}
  J_{eq} \dot{\omega}_m(t) = k_m i_m(t) + T_d.
\end{equation*}
Disturbance torques can be neglected by ensuring the motor spins without interruption in the laboratory.
The previous equation can be written in the Laplace domain as
\begin{equation}\label{eq:tf-load}
  sJ_{eq} \Omega_m(s) = k_m I_m(s).
\end{equation}
Substituting \eqref{eq:tf-load} into \eqref{eq:KVL-tf} yeilds
\begin{equation*}
  {\Omega_m(s) \over U(s)} = {k_m \over L_m J_{eq}s^2 + R_m J_{eq} s + {k_m}^2}.
\end{equation*}
For a small motor $R_m \gg L_m$, so the effect of $L_m$ can be neglected in our linear model.
The previous equation can be rewritten as
\begin{equation*}
  {\Omega_m(s) \over U(s)} = { 1 \over k_m \left( {J_{eq} R_m \over {k_m}^2} s + 1 \right)}.
\end{equation*}
Comparing the previous equation with the form of \eqref{eq:tf} implies
\begin{equation}\label{eq:static-params}
  K = {1 \over k_m} \quad \text{and} \quad \tau = {J_{eq} R_m \over {k_m}^2}.
\end{equation}
\cite{lab-manual} gives values for $J_m$, $M_l$ and $R_l$ which can be used to determine $J_{eq}$.
\begin{align*}
  J_{eq} &= J_m + {1 \over 2} M_l {R_l}^2 \\
         &= \left(\SI{11.6}{\gram\square\centi\meter}\right) + {1 \over 2} \left(\SI{68}{\gram}\right) \left(\SI{2.48}{\centi\meter}\right)^2 \\
         &= \SI{220.7}{\gram\square\centi\meter} \\
         &= \SI{2.207e-5}{\kilogram\square\meter}
\end{align*}
$J_{eq}$, $k_{m\>avg}$ and $R_{m\>avg}$ can be substituted into \eqref{eq:static-params} to yeild $K = \SI{17.55}{\radian\per\volt\per\second}$ and $\tau = \SI{0.106}{\second}$.
The prelab has $K = \SI{19.92}{\radian\per\volt\per\second}$ and $\tau = \SI{0.093}{\second}$.
\todo[inline]{Come up with a convincing explanation for this}

\subsection{Bump test}\label{sec:bump}
$K$ and $\tau$ can also be derived by only measuring the input $u_m$ and output $\omega_m$ during the application of a step input.
This technique is referred to as a ``bump test.''
Assuming the measured system has a first order response, the output will be
\begin{equation}\label{eq:dynamic-yt}
  y(t) = \Delta u K \left(1 - e^{{-t \over \tau}}\right).
\end{equation}
$K$ will be the ratio of steady state output to input
\begin{equation}\label{eq:dynamic-gain}
  K = {\Delta y \over \Delta u}.
\end{equation}
Substituting \eqref{eq:dynamic-gain} into \eqref{eq:dynamic-yt} at $t = \tau$ gives
\begin{equation}\label{eq:dynamic-tau}
  y(\tau) = \Delta y \left(1 - e^{-1}\right) \approx 0.63 \Delta y.
\end{equation}
The step input $u_m$ will be approximated by a square wave varying between \SI{1}{\volt} and \SI{5}{\volt} ($\Delta u = \SI{4}{\volt}$) and a frequency of \SI{0.4}{\hertz}.
Figure~\ref{fig:step} shows the response of the DC motor to this input.
\pagebreak
\begin{figure}[tbph]
  \centering
  \includegraphics[width=0.95\linewidth]{graphics/part3}
  \caption{Response of DC motor to step input}
  \label{fig:step}
\end{figure}

The response of the motor is $\Delta y = \SI{75}{\volt}$.
It reaches $0.63 \Delta y$ at $\Delta t = \SI{0.091}{\second} = \tau$. \eqref{eq:dynamic-gain} gives $K = \SI{18.75}{\radian\per\volt\per\second}$.
\todo[inline]{Come up with a convincing explanation for this}

\subsection{Model validation}\label{sec:validation}
QUCii will take values of $K$ and $\tau$ use them to generate an expected response to the measured input.
The square wave from Section~\ref{sec:bump} is reused as a step input.
Figure~\ref{fig:validation} shows models with parameters derived from the prelab, static analysis and bump test.
The model parameters in Figure~\ref{subfig:tuned} were determined be the best fit from manual parameter variation.
The expected response is in dark blue and the measured response is in red.
\begin{figure}[t!]
  \subfigure[Prelab: $K = \SI{19.92}{\radian\per\volt\per\second}$, $\tau = \SI{0.93}{\second}$]
  {
    \includegraphics[width=0.475\linewidth]{part4_prelab}
    \label{subfig:prelab}
  }
  \subfigure[Static analysis: $K = \SI{17.5}{\radian\per\volt\per\second}$, $\tau = \SI{0.106}{\second}$]
  {
    \includegraphics[width=0.475\linewidth]{part4_static}
    \label{subfig:static}
  }
  \subfigure[Bump test: $K = \SI{18.8}{\radian\per\volt\per\second}$, $\tau = \SI{0.091}{\second}$]
  {
    \includegraphics[width=0.475\linewidth]{part4_bump}
    \label{subfig:bump}
  }
  \subfigure[Tuned: $K = \SI{18.6}{\radian\per\volt\per\second}$, $\tau = \SI{0.090}{\second}$]
  {
    \includegraphics[width=0.475\linewidth]{part4_tuned}
    \label{subfig:tuned}
  }
  \caption{Performance of various models against measured motor response to step input}
  \label{fig:validation}
\end{figure}

The performance of the models is inversely correlated to the number of assumptions they make about the physical system at hand.
The prelab values of $R_m$ and $k_m$ did not precisely match those from Sections~\ref{sec:mres} and \ref{sec:mtorque} and its model is the furthest from the actual behavior of the motor.

The model derived from static analysis ignored the effects of $L_m$ and $T_d$ entirely and $i_m$ and $\omega_m$ when it was convenient.
The static analysis model performs considerably better than the prelab model but it underestimates the motor response at the upper steady state.

The bump test model performs extremely well.
It tracks the ``up'' response perfectly.
The model deviates very slightly on the ``down'' response, where it fails to track the observed decay of $\omega_m$ at steady state.
This deviance likely due to a breakdown of the linear model as $u_m \rightarrow \SI{0.4}{\volt}$, the voltage at which the motor overcomes the Coulomb friction and begins to rotate.
Observe the spike in $R_m$ in Table~\ref{table:Rm} at $u_m = \SI{1.00}{\volt}$.
This suggests Coulomb friction is significant over a range of values and manifests as non-linear behavior at low voltages. 

The tuned model is not a significant improvement over the bump test model.
This is because the non-linearities from Coulomb friction cannot be accounted for by varying $K$ and $\tau$ in a linear model.
