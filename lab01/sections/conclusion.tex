\section{Conclusion}\label{sec:conclusion}
In this experiment we were able to derive three different sets of parameters for the model of the DC motor.
The parameters derived from the prelab calculations did a poor job of approximating the motor's behavior because they make incorrect assumptions about the value of $R_m$ and $k_m$ of the motor.
Measurements of the motor in static conditions and parameter derivation from physical relations provided parameters for an adequate model.
However, the assumptions that $L_m$ and $T_d$ were non-existent result in a consistent underestimation of the motor's maximum angular velocity.
The parameters derived from the bump test, $K = \SI{18.8}{\radian\per\volt\per\second}$ and $\tau = \SI{0.091}{\second}$, almost perfectly model the behavior of the motor.
The model does not account for the decline in velocity at low voltages because it ignores the Coloumb friction in the motor which is greatest around \SI{\pm 0.4}{\volt} and negligible beyond \SI{2}{\volt}.
Manual tuning of $K$ and $\tau$ could not significantly improve on the accuracy of the bump test parameters.

This experiment did not investigate the model's behavior at the upper end of its validity: where a large enough $u_m$ is supplied so that the gain becomes saturated.
Future experiments with this motor must take into account the existence of this upper limit.
The model parameters we have derived will be most accurate when $u_m$ has an operational range of \SIrange{2}{5}{\volt}.

